%
% Copyright © 2017 Peeter Joot.  All Rights Reserved.
% Licenced as described in the file LICENSE under the root directory of this GIT repository.
%
One of the simplest curvilinear coordinate systems are cylindrical coordinates in a plane.  The parameterization associated with such a space is

\begin{equation}\label{eqn:2Dcylindrical:100}
\Bx(\rho, \phi) = \rho \Be_1 \exp\lr{ \Be_{12} \phi }.
\end{equation}

The curvilinear coordinate basis is therefore

\begin{subequations}
\label{eqn:2Dcylindrical:120}
\begin{equation}\label{eqn:2Dcylindrical:140}
\begin{aligned}
\Bx_\rho
&= \PD{\rho}{} \lr{ \rho \Be_1 \exp\lr{ \Be_{12} \phi } } \\
&= \Be_1 \exp\lr{ \Be_{12} \phi }
\end{aligned}
\end{equation}
\begin{equation}\label{eqn:2Dcylindrical:160}
\begin{aligned}
\Bx_\phi
&= \PD{\phi}{} \lr{ \rho \Be_1 \exp\lr{ \Be_{12} \phi } } \\
&= \rho \Be_1 \Be_{12} \exp\lr{ \Be_{12} \phi } \\
&= \rho \Be_2 \exp\lr{ \Be_{12} \phi }.
\end{aligned}
\end{equation}
\end{subequations}

Noting that this is a normal set of vectors, the reciprocal basis can be found by inspection

\begin{equation}\label{eqn:2Dcylindrical:180}
\begin{aligned}
\Bx^\rho &= \Be_1 \exp\lr{ \Be_{12} \phi } \\
\Bx^\phi &= \inv{\rho} \Be_2 \exp\lr{ \Be_{12} \phi }.
\end{aligned}
\end{equation}

For completeness, it's worth verifying that the gradient representation of the reciprocal frame provides this same result.  Those gradients can be computed by implicit differentiation of
\begin{equation}\label{eqn:2Dcylindrical:500}
\begin{aligned}
x^2 + y^2 &= \rho^2 \\
\tan\phi &= y/x,
\end{aligned}
\end{equation}

The derivatives are

\begin{equation}\label{eqn:2Dcylindrical:520}
\begin{aligned}
2 \rho \PD{x}{\rho} &= 2 x \\
2 \rho \PD{y}{\rho} &= 2 y \\
\inv{\cos^2\phi} \PD{x}{\phi} &= -\frac{y}{x^2} \\
\inv{\cos^2\phi} \PD{y}{\phi} &= \inv{x},
\end{aligned}
\end{equation}

The gradients are
\begin{subequations}
\label{eqn:2Dcylindrical:540}
\begin{equation}\label{eqn:2Dcylindrical:560}
\begin{aligned}
\spacegrad \rho
&= \inv{\rho} (\cos\phi, \sin\phi) \\
&= e_1 e^{\Be_{12} \phi} \\
&= \Bx^\rho
\end{aligned}
\end{equation}
\begin{equation}\label{eqn:2Dcylindrical:580}
\begin{aligned}
\spacegrad \phi
&= \cos^2 \phi \lr{ -\frac{y}{x^2}, \inv{x} } \\
&= \inv{\rho} ( -\sin\phi, \cos\phi ) \\
&= \frac{\Be_2}{\rho} ( \cos\phi + \Be_{12} \sin\phi ) \\
&= \frac{\Be_2}{\rho} e^{ \Be_{12} \phi } \\
&= \Bx^\phi,
\end{aligned}
\end{equation}
\end{subequations}

which is consistent with the result found by inspection as desired.

In this particular parmameterization, it is convieient to define a locally orthonormal coordinate basis \( \setlr{ \rhocap, \phicap } \)

\begin{equation}\label{eqn:2Dcylindrical:200}
\begin{aligned}
\rhocap &= \Bx_\rho = \Be_1 \exp\lr{ \Be_{12} \phi } \\
\phicap &= \inv{r} \Bx_\phi = \Be_2 \exp\lr{ \Be_{12} \phi },
\end{aligned}
\end{equation}

so that \( \Bx^\rho = \Bx_\rho = \rhocap \), \( \Bx_\phi = \rho \rhocap \), and \( \Bx^\phi = \rhocap/\rho \), and the gradient is

\begin{equation}\label{eqn:2Dcylindrical:600}
\begin{aligned}
\spacegrad
&= \Bx^\rho \PD{\rho}{} + \Bx^\phi \PD{\phi}{} \\
&= \rhocap \PD{\rho}{} +\inv{\rho} \phicap \PD{\phi}{}.
\end{aligned}
\end{equation}

The volume element for this subspace is
\begin{equation}\label{eqn:2Dcylindrical:220}
\begin{aligned}
d\Bx_\rho \wedge d\Bx_\phi
&= d\rho d\phi \Bx_\rho \wedge \Bx_\phi \\
&= d\rho d\phi \gpgradetwo{ \Bx_\rho \Bx_\phi } \\
&= d\rho d\phi \gpgradetwo{ \Be_1 \exp\lr{ \Be_{12} \phi } \rho \Be_2 \exp\lr{ \Be_{12} \phi } } \\
&= \rho d\rho d\phi \gpgradetwo{ \Be_1 \Be_2 \exp\lr{ -\Be_{12} \phi } \exp\lr{ \Be_{12} \phi } } \\
&= \rho d\rho d\phi \Be_{12}.
\end{aligned}
\end{equation}

Observe that the (oriented) volume of a circular region of radius \( r \) in this space has the expected result

\begin{equation}\label{eqn:2Dcylindrical:360}
\begin{aligned}
\int d\Bx_\rho \wedge d\Bx_\phi
&= \int_0^r \rho d\rho \int_0^{2\pi} d\phi \Be_{12} \\
&= \pi r^2 \Be_{12}.
\end{aligned}
\end{equation}

Given a vector \( \Bv = \Be_1 f(\rho, \phi) + \Be_2 g(\rho, \phi) \), the cylindrical representation \( \Bv = \Bv_\rho \rhocap + \Bv_\phi \phicap \) can be found by computing the dot products

\begin{subequations}
\label{eqn:2Dcylindrical:420}
\begin{equation}\label{eqn:2Dcylindrical:440}
\begin{aligned}
\Bv \cdot \rhocap
&= \gpgradezero{ (\Be_1 f + \Be_2 g) \Be_1 e^{\Be_{12} \phi} } \\
&= f \cos\phi + g \sin\phi
\end{aligned}
\end{equation}
\begin{equation}\label{eqn:2Dcylindrical:460}
\begin{aligned}
\Bv \cdot \phicap
&= \gpgradezero{ (\Be_1 f + \Be_2 g) \Be_2 e^{\Be_{12} \phi} } \\
&= g \cos\phi - f \sin\phi,
\end{aligned}
\end{equation}
\end{subequations}
so
\begin{equation}\label{eqn:2Dcylindrical:480}
\Bv = \lr{ f \cos\phi + g \sin\phi } \rhocap + \lr{ g \cos\phi - f \sin\phi } \phicap.
\end{equation}
