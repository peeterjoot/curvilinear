%
% Copyright © 2017 Peeter Joot.  All Rights Reserved.
% Licenced as described in the file LICENSE under the root directory of this GIT repository.
%
One of the simplest curvilinear coordinate systems are cylindrical coordinates in a plane.  The parameterization associated with such a space is

\begin{dmath}\label{eqn:2Dcylindrical:100}
\Bx(\rho, \theta) = \rho \Be_1 \exp\lr{ \Be_{12} \theta }.
\end{dmath}

The curvilinear coordinate basis is therefore

\begin{subequations}
\label{eqn:2Dcylindrical:120}
\begin{dmath}\label{eqn:2Dcylindrical:140}
\Bx_\rho
= \PD{\rho}{} \lr{ \rho \Be_1 \exp\lr{ \Be_{12} \theta } }
= \Be_1 \exp\lr{ \Be_{12} \theta }
\end{dmath}
\begin{dmath}\label{eqn:2Dcylindrical:160}
\Bx_\theta
= \PD{\theta}{} \lr{ \rho \Be_1 \exp\lr{ \Be_{12} \theta } }
= \rho
\Be_1 \Be_{12} \exp\lr{ \Be_{12} \theta }
= \rho
\Be_2 \exp\lr{ \Be_{12} \theta }.
\end{dmath}
\end{subequations}

Noting that this is a normal set of vectors, the reciprocal basis can be found by inspection

\begin{dmath}\label{eqn:2Dcylindrical:180}
\begin{aligned}
\Bx^\rho &= \Be_1 \exp\lr{ \Be_{12} \theta } \\
\Bx^\theta &= \inv{\rho} \Be_2 \exp\lr{ \Be_{12} \theta }.
\end{aligned}
\end{dmath}

In this particular parmameterization, it is convieient to define a locally orthonormal coordinate basis \( \setlr{ \rhocap, \thetacap } \)

\begin{dmath}\label{eqn:2Dcylindrical:200}
\begin{aligned}
\rhocap &= \Be_1 \exp\lr{ \Be_{12} \theta } \\
\thetacap &= \Be_2 \exp\lr{ \Be_{12} \theta },
\end{aligned}
\end{dmath}

so that \( \Bx^\rho = \Bx_\rho = \rhocap \), \( \Bx_\theta = \rho \rhocap \), and \( \Bx^\theta = \rhocap/\rho \).

The volume element for this subspace is
\begin{dmath}\label{eqn:2Dcylindrical:220}
d\Bx_\rho \wedge d\Bx_\theta
=
d\rho d\theta
\Bx_\rho \wedge \Bx_\theta
=
d\rho d\theta
\gpgradetwo{
\Bx_\rho \Bx_\theta
}
=
d\rho d\theta
\gpgradetwo{
\Be_1 \exp\lr{ \Be_{12} \theta } \rho
\Be_2 \exp\lr{ \Be_{12} \theta }
}
=
\rho d\rho d\theta
\gpgradetwo{
\Be_1 \Be_2 \exp\lr{ -\Be_{12} \theta }
\exp\lr{ \Be_{12} \theta }
}
=
\rho d\rho d\theta \Be_{12}.
\end{dmath}

Observe that the (oriented) volume of a circular region of radius \( r \) in this space has the expected result

\begin{dmath}\label{eqn:2Dcylindrical:360}
\int d\Bx_\rho \wedge d\Bx_\theta
=
\int_0^r \rho d\rho \int_0^{2\pi} d\theta \Be_{12}
= \pi r^2 \Be_{12}.
\end{dmath}

Given a vector \( \Bv = \Be_1 f(\rho, \theta) + \Be_2 g(\rho, \theta) \), the cylindrical representation \( \Bv = \Bv_\rho \rhocap + \Bv_\theta \thetacap \) can be found by computing the dot products

\begin{subequations}
\label{eqn:2Dcylindrical:420}
\begin{dmath}\label{eqn:2Dcylindrical:440}
\Bv \cdot \rhocap
=
\gpgradezero{ (\Be_1 f + \Be_2 g) \Be_1 e^{\Be_{12} \theta} }
=
f \cos\theta + g \sin\theta
\end{dmath}
\begin{dmath}\label{eqn:2Dcylindrical:460}
\Bv \cdot \thetacap
=
\gpgradezero{ (\Be_1 f + \Be_2 g) \Be_2 e^{\Be_{12} \theta} }
=
g \cos\theta - f \sin\theta,
\end{dmath}
\end{subequations}

so
\begin{dmath}\label{eqn:2Dcylindrical:480}
\Bv = \lr{ f \cos\theta + g \sin\theta } \rhocap + \lr{ g \cos\theta - f \sin\theta } \thetacap.
\end{dmath}
