%
% Copyright © 2017 Peeter Joot.  All Rights Reserved.
% Licenced as described in the file LICENSE under the root directory of this GIT repository.
%
%{

The cylindrical unit vectors can be expressed in Pauli matrices trivially.  From \cref{eqn:2Dcylindrical:200}

\begin{subequations}
\label{eqn:pauliRepresentation:1}
\begin{dmath}\label{eqn:pauliRepresentation:2}
\Bsigma \cdot \rhocap
=
\Bsigma \cdot \lr{ \Be_1 \exp\lr{ \Be_{12} \theta } }
=
\Bsigma \cdot \lr{ \Be_1 \lr{ \cos\theta + \Be_1 \Be_2 \sin\theta } }
=
\Bsigma \cdot \lr{ \Be_1 \cos\theta + \Be_2 \sin\theta }
=
\sigma_1 \cos\theta + \sigma_2 \sin\theta
= 
\PauliX \cos\theta + \PauliY \sin\theta
=
\begin{bmatrix}
0 & \cos\theta - i \sin\theta \\
\cos\theta + i \sin\theta & 0
\end{bmatrix}
=
\begin{bmatrix}
0 & e^{-i\theta} \\
e^{i\theta} & 0
\end{bmatrix}
.
\end{dmath}
\begin{dmath}\label{eqn:pauliRepresentation:3}
\Bsigma \cdot
\thetacap 
= 
\Bsigma \cdot \lr{ \Be_2 \exp\lr{ \Be_{12} \theta } }
=
\Bsigma \cdot \lr{ \Be_2 \lr{ \cos\theta + \Be_1 \Be_2 \sin\theta } }
= 
\Bsigma \cdot \lr{ \Be_2 \cos\theta - \Be_1 \sin\theta }
= 
\sigma_2 \cos\theta - \sigma_1 \sin\theta
= 
\PauliY \cos\theta - \PauliX \sin\theta
=
\begin{bmatrix}
0 & -i(\cos\theta - i\sin\theta) \\
i(\cos\theta + i\sin\theta) & 0
\end{bmatrix}
=
i 
\begin{bmatrix}
0 & -e^{-i\theta} \\
e^{i\theta} & 0
\end{bmatrix}
\end{dmath}
\end{subequations}

Notice that translating any expression from GA notation really just requires a substitution \( \Be_k \rightarrow \sigma_k \), and the CAS implementation should be do the grunt work of such a translation.

We have therefore seen that in  cylindrical coordinates a different set of Pauli matrices can be established: (I have switched the theta to phi)

\begin{subequations}
\label{eqn:pauliRepresentation:1}
\begin{dmath}\label{eqn:pauliRepresentation:cyl1}
\sigma_{\rho} =
\begin{bmatrix}
0 & e^{-i\phi} \\
e^{i\phi} & 0
\end{bmatrix}
\end{dmath}
%
\begin{dmath}\label{eqn:pauliRepresentation:cyl2}
\sigma_{\phi}
=
\begin{bmatrix}
0 & - i \, e^{-i\phi} \\
i \,e^{i\phi} & 0
\end{bmatrix}
\end{dmath}
%
\begin{dmath}\label{eqn:pauliRepresentation:cyl3}
\sigma_{z}
=
\begin{bmatrix}
1 & 0 \\
0 & -1
\end{bmatrix}
\end{dmath}
\end{subequations}


If the CAS system being used can handle functions of matrices, there might not be a need to explicitly expand the rotation operator \( e^{\Be_{12} \phi } \) explicitly, as done above.

When dealing with a vector $\BA= A_{\rho}\rhocap + A_{\phi}\hat{\phi}  + A_z \hat{\Bz}$, in the circular cylindrical coordinate system, the relative Pauli matrix is: 
\begin{dmath}\label{eqn:pauliRepresentation:3a}
\tilde{A }=
\begin{bmatrix}
A_z & e^{-i\phi}(A_{\rho} - i A_{\phi}) \\
e^{i\phi}(A_{\rho} + i A_{\phi}) & -A_z
\end{bmatrix} \, .
\end{dmath}

The nabla operator in cylindrical coordinates system can be expressed in terms of Pauli matrices as
%
\begin{dmath}\label{eqn:pauliRepresentation:3b}
\tilde{\nabla} =
\begin{bmatrix}
\partial_z & e^{-i\phi}(\partial_{\rho} - \frac {i}{\rho} \partial_{\phi}) \\
e^{i\phi}(\partial_{\rho} + \frac {i}{\rho} \partial_{\phi}) & -\partial_z
\end{bmatrix} \, .
\end{dmath}
%
It is important to note that in the nabla operator the term $e^{\pm i\phi}$ is before the derivatives, while in (\ref{eqn:pauliRepresentation:3a}) this term has to be differentiated by the nabla operator.


For the spherical coordinate system, by introducing the definitions $c_{\theta}= \cos(\theta)$  and $s_{\theta}= \sin(\theta)$ we have:
%
\begin{subequations}
\label{eqn:pauliRepresentationsp:1}
\begin{dmath}\label{eqn:pauliRepresentation:sph1}
\sigma_{r} =
\begin{bmatrix}
c_{\theta} & s_{\theta} \, e^{-i\phi} \\
s_{\theta}\,e^{i\phi} & -c_{\theta} \end{bmatrix}
\end{dmath}
%
\begin{dmath}\label{eqn:pauliRepresentation:sph2}
\sigma_{\theta}
=
\begin{bmatrix}
-s_{\theta} &  c_{\theta} \, e^{-i\phi} \\
c_{\theta} \,e^{i\phi} & s_{\theta}
\end{bmatrix}
\end{dmath}
%
\begin{dmath}\label{eqn:pauliRepresentation:sph3}
\sigma_{\phi}
=
\begin{bmatrix}
0 & - i \, e^{-i\phi} \\
i \,e^{i\phi} & 0
\end{bmatrix}
\end{dmath}
%
\end{subequations}

Accordingly the nabla operator becomes...
%}
