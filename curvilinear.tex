%
% Copyright � 2017 Peeter Joot.  All Rights Reserved.
% Licenced as described in the file LICENSE under the root directory of this GIT repository.
%
%{
\input{../latex/blogpost.tex}
\renewcommand{\basename}{curvilinear}
%\renewcommand{\dirname}{notes/phy1520/}
\renewcommand{\dirname}{notes/ece1228-electromagnetic-theory/}
%\newcommand{\dateintitle}{}
%\newcommand{\keywords}{}

\input{../latex/peeter_prologue_print2.tex}

\usepackage{peeters_layout_exercise}
\usepackage{peeters_braket}
\usepackage{peeters_figures}
\usepackage{siunitx}
%\usepackage{mhchem} % \ce{}
%\usepackage{macros_bm} % \bcM
%\usepackage{macros_qed} % \qedmarker
%\usepackage{txfonts} % \ointclockwise

\beginArtNoToc

\generatetitle{Curvilinear coordinates}
%\chapter{Curvilinear coordinates}
%\label{chap:curvilinear}

%
% Copyright © 2017 Peeter Joot.  All Rights Reserved.
% Licenced as described in the file LICENSE under the root directory of this GIT repository.
%
Curvilinear coordinates can be defined for any subspace spanned by a parameterized vector into that space.  Given, for example, a vector into a subspace parameterized by parameters \(u,v\)

\begin{dmath}\label{eqn:curvilinear:20}
\Bx = \Bx(u, v),
\end{dmath}

the partials with respect to these parameters

\begin{dmath}\label{eqn:curvilinear:40}
\begin{aligned}
d\Bx_u &= \PD{u}{\Bx} du \\
d\Bx_v &= \PD{v}{\Bx} dv
\end{aligned}
\end{dmath}

span the space at the point that these partials are evaluated.  In the language of differential forms, this localized subspace is called the tangent space.  It is generally desirable to consider parameterizations for which the tangent space volume element is non-zero.  In this case, that is

\begin{dmath}\label{eqn:curvilinear:60}
d\Bx_u \wedge d\Bx_v \ne 0.
\end{dmath}

The differentials form a basis for the tangent space, as do the partials themselves

\begin{dmath}\label{eqn:curvilinear:80}
\begin{aligned}
\Bx_u &= \PD{u}{\Bx} \\
\Bx_v &= \PD{v}{\Bx}.
\end{aligned}
\end{dmath}

There is no reason to presume that there is any orthonormality constraint on the basis \( \setlr{ \Bx_u, \Bx_v } \) for this two parameter subspace, so a reciprocal basis \( \setlr{ \Bx^u, \Bx^v } \)
must be used to compute coordinates.
%, defined by \( \Bx^i \cdot \Bx_j = {\delta^i}_j \),  must be used to compute coordinates.   [SEE: GAelectrodynamics: 3.1].

More generally, given a parameterization of \( \Bx(u_1, u_2, \cdots, u_k) \), a curvilinear basis defined on the tangent space is induced by the partials

\begin{dmath}\label{eqn:curvilinear:240}
\Bx_{u_i} = \PD{u_i}{\Bx}.
\end{dmath}

The volume element for the subspace is

\begin{dmath}\label{eqn:curvilinear:260}
d^k \Bx = du_1 du_2 \cdots du_k\,
\Bx_{u_1} \wedge
\Bx_{u_2} \wedge \cdots \wedge
\Bx_{u_k}.
\end{dmath}

Unlike a scalar volume, this volume element is oriented.  Any multivector can be expressed in terms of the curvilinear basis \( \setlr{ \Bx_{u_1}, \Bx_{u_2}, \cdots, \Bx_k} \), but computation of the curvilinear coordinates requires the reciprocal basis.  For example, a vector \( \Bf \) constrained to the tangent space admits a representation

\begin{dmath}\label{eqn:curvilinear:380}
\Bf = \sum_i a_i \Bx_{u_i}.
\end{dmath}

Dotting with \( \Bx^{u_j} \) gives

\begin{dmath}\label{eqn:curvilinear:280}
\Bf \cdot \Bx^{u_j}
= \sum_i a_i \Bx_{u_i} \cdot \Bx^{u_j}
= \sum_i a_i {\delta^i}_j
= a_j,
\end{dmath}

so
\begin{dmath}\label{eqn:curvilinear:300}
\Bf = \sum_i \lr{ \Bf \cdot \Bx^{u_i} } \Bx_{u_i}.
\end{dmath}

Higher grade multivector objects may also be represented in curvilinear coordinates.  For example, given a bivector constrained to the tangent space

\begin{dmath}\label{eqn:curvilinear:320}
B = \inv{2} \sum_{i, j} b_{ij} \Bx_{u_i} \wedge \Bx_{u_j},
\end{dmath}

the coordinates \( b_{ij} \) can be determined by dotting \( B \) with \( \Bx^{u_j} \wedge \Bx^{u_i} \), yielding

\begin{dmath}\label{eqn:curvilinear:340}
B \cdot \lr{ \Bx^{u_j} \wedge \Bx^{u_i} }
=
\inv{2} \sum_{i' , j'} b_{i'j'} \lr{ \Bx_{u_i'} \wedge \Bx_{u_j'} } \cdot \lr{ \Bx^{u_j} \wedge \Bx^{u_i} }
=
\inv{2} \sum_{i' , j'} b_{i'j'} \lr{ \lr{ \Bx_{u_i'} \wedge \Bx_{u_j'} } \cdot \Bx^{u_j} } \cdot \Bx^{u_i}
=
\inv{2} \sum_{i' , j'} b_{i'j'} \lr{  \Bx_{u_i'} {\delta_j'}^j - \Bx_{u_j'} {\delta_i'}^j } \cdot \Bx^{u_i}
=
\inv{2} \sum_{i' , j'} b_{i'j'} \lr{  {\delta_i'}^i {\delta_j'}^j - {\delta_j'}^i {\delta_i'}^j }
=
\inv{2} \lr{ b_{i j} - b_{ji} }.
\end{dmath}

When \( i \ne j \) this is \( b_{ij} \) and is zero otherwise.  The curvilinear representation of the bivector is therefore

\begin{dmath}\label{eqn:curvilinear:400}
B = \sum_{i < j} \lr{ B \cdot \lr{ \Bx^{u_j} \wedge \Bx^{u_i} }} \Bx_{u_i} \wedge \Bx_{u_j}.
\end{dmath}



%
% Copyright � 2017 Peeter Joot.  All Rights Reserved.
% Licenced as described in the file LICENSE under the root directory of this GIT repository.
%
While the reciprocal frame can be computed explcitly, it can also be computed very simply by computing the gradient of the parameters themselves.  Two theorems relate the gradient and the reciprocal frame vectors.

\maketheorem{Gradient definition of reciprocal frame vectors}{thm:curvilinearGradient:1}{

Given a curvilinear basis \( \setlr{ \Bx_k } \), the reciprocal frame vectors are

\begin{equation*}
\Bx^i = \spacegrad u_i.
\end{equation*}
} % theorem

\maketheorem{Curvilinear representation of the gradient}{thm:curvilinearGradient:2}{

Given an n-parameter representation of a vector that spans an n-dimensional space

\begin{equation*}
\Bx = \Bx(u_1, \cdots, u_n),
\end{equation*}

the curvilinear representation of the gradient is

\begin{equation*}
\spacegrad = \sum_i \Bx^i \PD{u_i}{}.
\end{equation*}

It is often convienent to write this as

\begin{equation*}
\spacegrad = \sum_{i=1}^n \Bx^i \partial_i,
\end{equation*}

or the same with sums over mixed indexes implied.

} % theorem

The proof of both are both just applications of the chain rule.  Assuming \cref{thm:curvilinearGradient:1} to be true, then the dot products of the reciprocal frame vectors with the curvilinear basis vectors are

\begin{equation}\label{eqn:curvilinearGradient:20}
\begin{aligned}
\Bx^i \cdot \Bx_j
&= (\spacegrad u_i) \cdot \PD{u_j}{\Bx} \\
&= \sum_{r,s=1}^n \lr{ \Be_r \PD{x_r}{u_i} } \cdot \lr{ \Be_s \PD{u_j}{x_s} } \\
&= \sum_{r,s=1}^n (\Be_r \cdot \Be_s) \PD{x_r}{u_i} \PD{u_j}{x_s} \\
&= \sum_{r,s=1}^n \delta_{rs} \PD{x_r}{u_i} \PD{u_j}{x_s} \\
&= \sum_{r=1}^n \PD{x_r}{u_i} \PD{u_j}{x_r} \\
&= \PD{u_i}{u_j} \\
&= \delta_{ij}.
\end{aligned}
\end{equation}

This shows that \( \Bx^i = \spacegrad u_i \) has the properties required of the reciprocal frame, proving the theorem.

The curvilinear representation of the gradient follows from the gradient representation of the reciprocal frame, and the chain rule.  The sum in \cref{thm:curvilinearGradient:2} expands as

\begin{equation}\label{eqn:curvilinearGradient:40}
\begin{aligned}
\sum_{i=1}^n
\Bx^i \PD{u_i}{F}
&=
\sum_{i=1}^n
(\spacegrad u_i) \PD{u_i}{F} \\
&=
\sum_{i,j=1}^n
\Be_j \PD{x_j}{u_i}
\PD{u_i}{F} \\
&=
\sum_{j=1}^n
\Be_j
\PD{x_j}{F} \\
&=
\spacegrad F,
\end{aligned}
\end{equation}
which proves the result.

Note that the gradient representation of the reciprocal frame is mainly useful for theoretical reasons (i.e. the proof of the curvilinear representation of the gradient).  In many cases it will likely be more difficult to compute the reciprocal frame vectors using the gradient of the parameters than

An excellent (and more detailed) discussion of the relationships of the reciprocal frame and the gradient can be found in \citep{aMacdonaldVAGC}.


\section{Examples}

Many of the abstract concepts above are illuminated nicely by considering some examples.

\makeexample{2D Cylindrical coordinates}{example:curvilinear:1}{
%
% Copyright © 2017 Peeter Joot.  All Rights Reserved.
% Licenced as described in the file LICENSE under the root directory of this GIT repository.
%
One of the simplest curvilinear coordinate systems are cylindrical coordinates in a plane.  The parameterization associated with such a space is

\begin{equation}\label{eqn:2Dcylindrical:100}
\Bx(\rho, \phi) = \rho \Be_1 \exp\lr{ \Be_{12} \phi }.
\end{equation}

The curvilinear coordinate basis is therefore

\begin{subequations}
\label{eqn:2Dcylindrical:120}
\begin{equation}\label{eqn:2Dcylindrical:140}
\begin{aligned}
\Bx_\rho
&= \PD{\rho}{} \lr{ \rho \Be_1 \exp\lr{ \Be_{12} \phi } } \\
&= \Be_1 \exp\lr{ \Be_{12} \phi }
\end{aligned}
\end{equation}
\begin{equation}\label{eqn:2Dcylindrical:160}
\begin{aligned}
\Bx_\phi
&= \PD{\phi}{} \lr{ \rho \Be_1 \exp\lr{ \Be_{12} \phi } } \\
&= \rho \Be_1 \Be_{12} \exp\lr{ \Be_{12} \phi } \\
&= \rho \Be_2 \exp\lr{ \Be_{12} \phi }.
\end{aligned}
\end{equation}
\end{subequations}

Noting that this is a normal set of vectors, the reciprocal basis can be found by inspection

\begin{equation}\label{eqn:2Dcylindrical:180}
\begin{aligned}
\Bx^\rho &= \Be_1 \exp\lr{ \Be_{12} \phi } \\
\Bx^\phi &= \inv{\rho} \Be_2 \exp\lr{ \Be_{12} \phi }.
\end{aligned}
\end{equation}

For completeness, it's worth verifying that the gradient representation of the reciprocal frame provides this same result.  Those gradients can be computed by implicit differentiation of
\begin{equation}\label{eqn:2Dcylindrical:500}
\begin{aligned}
x^2 + y^2 &= \rho^2 \\
\tan\phi &= y/x,
\end{aligned}
\end{equation}

The derivatives are

\begin{equation}\label{eqn:2Dcylindrical:520}
\begin{aligned}
2 \rho \PD{x}{\rho} &= 2 x \\
2 \rho \PD{y}{\rho} &= 2 y \\
\inv{\cos^2\phi} \PD{x}{\phi} &= -\frac{y}{x^2} \\
\inv{\cos^2\phi} \PD{y}{\phi} &= \inv{x},
\end{aligned}
\end{equation}

The gradients are
\begin{subequations}
\label{eqn:2Dcylindrical:540}
\begin{equation}\label{eqn:2Dcylindrical:560}
\begin{aligned}
\spacegrad \rho
&= \inv{\rho} (\cos\phi, \sin\phi) \\
&= e_1 e^{\Be_{12} \phi} \\
&= \Bx^\rho
\end{aligned}
\end{equation}
\begin{equation}\label{eqn:2Dcylindrical:580}
\begin{aligned}
\spacegrad \phi
&= \cos^2 \phi \lr{ -\frac{y}{x^2}, \inv{x} } \\
&= \inv{\rho} ( -\sin\phi, \cos\phi ) \\
&= \frac{\Be_2}{\rho} ( \cos\phi + \Be_{12} \sin\phi ) \\
&= \frac{\Be_2}{\rho} e^{ \Be_{12} \phi } \\
&= \Bx^\phi,
\end{aligned}
\end{equation}
\end{subequations}

which is consistent with the result found by inspection as desired.

In this particular parmameterization, it is convieient to define a locally orthonormal coordinate basis \( \setlr{ \rhocap, \phicap } \)

\begin{equation}\label{eqn:2Dcylindrical:200}
\begin{aligned}
\rhocap &= \Bx_\rho = \Be_1 \exp\lr{ \Be_{12} \phi } \\
\phicap &= \inv{r} \Bx_\phi = \Be_2 \exp\lr{ \Be_{12} \phi },
\end{aligned}
\end{equation}

so that \( \Bx^\rho = \Bx_\rho = \rhocap \), \( \Bx_\phi = \rho \rhocap \), and \( \Bx^\phi = \rhocap/\rho \), and the gradient is

\begin{equation}\label{eqn:2Dcylindrical:600}
\begin{aligned}
\spacegrad
&= \Bx^\rho \PD{\rho}{} + \Bx^\phi \PD{\phi}{} \\
&= \rhocap \PD{\rho}{} +\inv{\rho} \phicap \PD{\phi}{}.
\end{aligned}
\end{equation}

The volume element for this subspace is
\begin{equation}\label{eqn:2Dcylindrical:220}
\begin{aligned}
d\Bx_\rho \wedge d\Bx_\phi
&= d\rho d\phi \Bx_\rho \wedge \Bx_\phi \\
&= d\rho d\phi \gpgradetwo{ \Bx_\rho \Bx_\phi } \\
&= d\rho d\phi \gpgradetwo{ \Be_1 \exp\lr{ \Be_{12} \phi } \rho \Be_2 \exp\lr{ \Be_{12} \phi } } \\
&= \rho d\rho d\phi \gpgradetwo{ \Be_1 \Be_2 \exp\lr{ -\Be_{12} \phi } \exp\lr{ \Be_{12} \phi } } \\
&= \rho d\rho d\phi \Be_{12}.
\end{aligned}
\end{equation}

Observe that the (oriented) volume of a circular region of radius \( r \) in this space has the expected result

\begin{equation}\label{eqn:2Dcylindrical:360}
\begin{aligned}
\int d\Bx_\rho \wedge d\Bx_\phi
&= \int_0^r \rho d\rho \int_0^{2\pi} d\phi \Be_{12} \\
&= \pi r^2 \Be_{12}.
\end{aligned}
\end{equation}

Given a vector \( \Bv = \Be_1 f(\rho, \phi) + \Be_2 g(\rho, \phi) \), the cylindrical representation \( \Bv = \Bv_\rho \rhocap + \Bv_\phi \phicap \) can be found by computing the dot products

\begin{subequations}
\label{eqn:2Dcylindrical:420}
\begin{equation}\label{eqn:2Dcylindrical:440}
\begin{aligned}
\Bv \cdot \rhocap
&= \gpgradezero{ (\Be_1 f + \Be_2 g) \Be_1 e^{\Be_{12} \phi} } \\
&= f \cos\phi + g \sin\phi
\end{aligned}
\end{equation}
\begin{equation}\label{eqn:2Dcylindrical:460}
\begin{aligned}
\Bv \cdot \phicap
&= \gpgradezero{ (\Be_1 f + \Be_2 g) \Be_2 e^{\Be_{12} \phi} } \\
&= g \cos\phi - f \sin\phi,
\end{aligned}
\end{equation}
\end{subequations}
so
\begin{equation}\label{eqn:2Dcylindrical:480}
\Bv = \lr{ f \cos\phi + g \sin\phi } \rhocap + \lr{ g \cos\phi - f \sin\phi } \phicap.
\end{equation}

} % example

\section{Pauli representation}

%
% Copyright © 2017 Peeter Joot.  All Rights Reserved.
% Licenced as described in the file LICENSE under the root directory of this GIT repository.
%
%{

The cylindrical unit vectors can be expressed in Pauli matrices trivially.  From \cref{eqn:2Dcylindrical:200}

\begin{subequations}
\label{eqn:pauliRepresentation:1}
\begin{dmath}\label{eqn:pauliRepresentation:2}
\Bsigma \cdot \rhocap
=
\Bsigma \cdot \lr{ \Be_1 \exp\lr{ \Be_{12} \theta } }
=
\Bsigma \cdot \lr{ \Be_1 \lr{ \cos\theta + \Be_1 \Be_2 \sin\theta } }
=
\Bsigma \cdot \lr{ \Be_1 \cos\theta + \Be_2 \sin\theta }
=
\sigma_1 \cos\theta + \sigma_2 \sin\theta
= 
\PauliX \cos\theta + \PauliY \sin\theta
=
\begin{bmatrix}
0 & \cos\theta - i \sin\theta \\
\cos\theta + i \sin\theta & 0
\end{bmatrix}
=
\begin{bmatrix}
0 & e^{-i\theta} \\
e^{i\theta} & 0
\end{bmatrix}
.
\end{dmath}
\begin{dmath}\label{eqn:pauliRepresentation:3}
\Bsigma \cdot
\thetacap 
= 
\Bsigma \cdot \lr{ \Be_2 \exp\lr{ \Be_{12} \theta } }
=
\Bsigma \cdot \lr{ \Be_2 \lr{ \cos\theta + \Be_1 \Be_2 \sin\theta } }
= 
\Bsigma \cdot \lr{ \Be_2 \cos\theta - \Be_1 \sin\theta }
= 
\sigma_2 \cos\theta - \sigma_1 \sin\theta
= 
\PauliY \cos\theta - \PauliX \sin\theta
=
\begin{bmatrix}
0 & -i(\cos\theta - i\sin\theta) \\
i(\cos\theta + i\sin\theta) & 0
\end{bmatrix}
=
i 
\begin{bmatrix}
0 & -e^{-i\theta} \\
e^{i\theta} & 0
\end{bmatrix}
\end{dmath}
\end{subequations}

Notice that translating any expression from GA notation really just requires a substitution \( \Be_k \rightarrow \sigma_k \), and the CAS implementation should be do the grunt work of such a translation.

We have therefore seen that in  cylindrical coordinates a different set of Pauli matrices can be established: (I have switched the theta to phi)

\begin{subequations}
\label{eqn:pauliRepresentation:1}
\begin{dmath}\label{eqn:pauliRepresentation:cyl1}
\sigma_{\rho} =
\begin{bmatrix}
0 & e^{-i\phi} \\
e^{i\phi} & 0
\end{bmatrix}
\end{dmath}
%
\begin{dmath}\label{eqn:pauliRepresentation:cyl2}
\sigma_{\phi}
=
\begin{bmatrix}
0 & - i \, e^{-i\phi} \\
i \,e^{i\phi} & 0
\end{bmatrix}
\end{dmath}
%
\begin{dmath}\label{eqn:pauliRepresentation:cyl3}
\sigma_{z}
=
\begin{bmatrix}
1 & 0 \\
0 & -1
\end{bmatrix}
\end{dmath}
\end{subequations}


If the CAS system being used can handle functions of matrices, there might not be a need to explicitly expand the rotation operator \( e^{\Be_{12} \phi } \) explicitly, as done above.

When dealing with a vector $\BA= A_{\rho}\rhocap + A_{\phi}\hat{\phi}  + A_z \hat{\Bz}$, in the circular cylindrical coordinate system, the relative Pauli matrix is: 
\begin{dmath}\label{eqn:pauliRepresentation:3a}
\tilde{A }=
\begin{bmatrix}
A_z & e^{-i\phi}(A_{\rho} - i A_{\phi}) \\
e^{i\phi}(A_{\rho} + i A_{\phi}) & -A_z
\end{bmatrix} \, .
\end{dmath}

The nabla operator in cylindrical coordinates system can be expressed in terms of Pauli matrices as
%
\begin{dmath}\label{eqn:pauliRepresentation:3b}
\tilde{\nabla} =
\begin{bmatrix}
\partial_z & e^{-i\phi}(\partial_{\rho} - \frac {i}{\rho} \partial_{\phi}) \\
e^{i\phi}(\partial_{\rho} + \frac {i}{\rho} \partial_{\phi}) & -\partial_z
\end{bmatrix} \, .
\end{dmath}
%
It is important to note that in the nabla operator the term $e^{\pm i\phi}$ is before the derivatives, while in (\ref{eqn:pauliRepresentation:3a}) this term has to be differentiated by the nabla operator.


For the spherical coordinate system, by introducing the definitions $c_{\theta}= \cos(\theta)$  and $s_{\theta}= \sin(\theta)$ we have:
%
\begin{subequations}
\label{eqn:pauliRepresentationsp:1}
\begin{dmath}\label{eqn:pauliRepresentation:sph1}
\sigma_{r} =
\begin{bmatrix}
c_{\theta} & s_{\theta} \, e^{-i\phi} \\
s_{\theta}\,e^{i\phi} & -c_{\theta} \end{bmatrix}
\end{dmath}
%
\begin{dmath}\label{eqn:pauliRepresentation:sph2}
\sigma_{\theta}
=
\begin{bmatrix}
-s_{\theta} &  c_{\theta} \, e^{-i\phi} \\
c_{\theta} \,e^{i\phi} & s_{\theta}
\end{bmatrix}
\end{dmath}
%
\begin{dmath}\label{eqn:pauliRepresentation:sph3}
\sigma_{\phi}
=
\begin{bmatrix}
0 & - i \, e^{-i\phi} \\
i \,e^{i\phi} & 0
\end{bmatrix}
\end{dmath}
%
\end{subequations}

Accordingly the nabla operator becomes...
%}



%}
\EndArticle
